\documentclass{article}
\usepackage[utf8]{inputenc}
\usepackage[version=4]{mhchem} % símbolos químicos
\usepackage{tabularx}
\usepackage{booktabs}

\usepackage{array} % gestión de tablas
\usepackage{xcolor}
\usepackage{tcolorbox}
\usepackage{graphicx}

\graphicspath{ {img/}}

\title{Unit 5. Chemical Reactions}
\author{Francisco Javier Guijarro}

\begin{document}

	\maketitle
	
	\section{Changes in matter}
	
		A change is a variation in the properties of an object. \textbf{Matter} to change requires \textbf{energy}.
	
	
		Changes can be physical or chemical. 
	
		\begin{itemize}
			\item In \textbf{physical} changes, matter change without be converted in a new substance, because the chemical composition stays the same. Usually, physical changes are \textbf{reversible}.
			\item In \textbf{chemical} changes, matter is converted in a new substance, because there is a change in the chemical composition. Most chemical changes are \textbf{irreversible}.

		\end{itemize}
	
	\section{Chemical reaction characteristics}
	
		\begin{enumerate}
			\item One or more substances are converted into new substances.
				\begin{itemize}
					\item \textbf{Reactants} are the initial substances.
					\item \textbf{Products} are the new formed substances. \par 
					\ce{Reactants -> Products}
				\end{itemize}  
			\item During a chemical reaction energy is released or absorbed.
				\begin{itemize}
					\item Reactions that release energy are called \textbf{exothermic}
					\item Reactions that absorb energy are called \textbf{endothermic}
				\end{itemize}
			\item \textbf{Law of the conservation of mass}. In a chemical reaction the total mass of the reactants is equal to the total mass of the products
			\item The \textbf{speed} of a reaction is the \textbf{rate} at which reactants are transformed into products.
				
		\end{enumerate}
		
		\subsection*{Example}
		
			56 g of iron (Fe) react with 32 g of sulphur (S) to make 88 g of iron sulphide. Check:
			\begin{enumerate}
				\item The law of conservation of mass
				\item If 28 g of iron react with 16 g of sulphur, how much iron is obtained?
				\item How much sulphur is needed to react with 42 g of iron if 66 g of iron sulphide is obtained?
			\end{enumerate}
		
			\begin{center}
				\ce{Fe + S ->[heat] FeS}
			\end{center}
		
			\begin{table}[htp]
				\centering
					\begin{tabular}{@{}ccc@{}}
						\toprule
 							mass (g) of Fe & mass (g) of S & mass (g) of FeS \\ \midrule
 							56 & 32  & 88 \\
 							28 & 16  & ? \\
 							42 & ? & 66 \\ \bottomrule
					\end{tabular}
			\end{table}
			
			\subsubsection*{Explaination}
				\begin{enumerate}
					\item Check if the law of conservation of mass is obeyed
					\[ 56\ g + 32\ g = 88\ g \]
					\item Add the mass of reactants
					\[ 28\ g + 16\ g = 44\ g \]
					\item Apply the law of conservation of mass
					\[ 42 + x = 66 \Longrightarrow x = 66 - 42 = 24\ g  \]
				\end{enumerate}

		
	
	
	\section{Atomic interpretation of a chemical reaction}
	
	\section{Chemical equations}
	
	\section{Basic laws of chemical reactions}
	
	\section{Chemistry in society}

\end{document}